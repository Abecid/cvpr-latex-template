\begin{abstract}
% Abstract goes here.
Appearance-based gaze estimation has been very successful with the use of deep learning. And there have been many works that improved domain generalization for gaze estimation. However, even though there have been much progress in domain generalization for gaze estimation, most of the recent work have been focused on cross-dataset performance-- accounting for different distributions in illuminations, head pose, and lighting. Although improving gaze estimation in different distributions with RGB images is important, near-infrared image based gaze estimation is also critical for gaze estimation in the dark. Also there are inherent limitations relying solely on supervised learning for regression tasks. This paper contributes to solving these problems and proposes GazeCWL, a novel framework for gaze estimation with near-infrared images using contrastive learning. This leverages a adversarial attack technique for data augmentation and a novel contrastive loss function specifically for regression tasks that effectively clusters the features of different samples in the latent space. Our model outperforms previous domain generalization models in infrared image based gaze estimation and outperforms the baseline by 45.6\% while improving the state-of-the-art by 8.6\% we demonstrate the efficacy of our method. 
\end{abstract}